We will use Fluidsim\cite{fluiddyn}, which is a HPC\footnote{High-Performance Computing} code using pseudospectral solvers.

\section{Numerical setup}
We will perform computations both on personal computer, for the small grid in order to test the scripts we write, and on a cluster host by GRICAD, which will allows us to simulate grid up to $1024^2 \times 256$. We need to setup the computing environment either with GUIX (which doesn't work in my case as of Friday 20 December) or conda to manage librairies. 

\section{Simulation parameters}

We want to perform a DNS\footnote{Direct Numerical Simulation}, this means that we need to have a resolution smaller than the Kolmogorov length scale $\eta$. We will use pseudospectal computing method which means that we must have choose the Reynolds of the simulation such as $k_{max} \eta \geqslant 1$ where $k_{max} = \frac{2\pi}{L} \frac{n_x}{2}$, we also know that $\frac{L}{\eta} = Re^{3/4}$ with this, we can express the Reynolds as a function of our grid parameter $n_x$
\begin{equation}
	Re = \left( \pi n_x \right)^{4/3}
\end{equation}