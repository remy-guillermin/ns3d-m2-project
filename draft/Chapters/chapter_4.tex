We will use Fluidsim\cite{fluiddyn}, which is a HPC\footnote{High-Performance Computing} code using pseudospectral solvers.

\section{Numerical setup}
We will perform computations both on personal computer, for the small grid in order to test the scripts we write, and on a cluster host by GRICAD, which will allows us to simulate grid up to $1024^2 \times 256$. We need to setup the computing environment either with GUIX (which doesn't work in my case as of Friday 20 December) or conda to manage librairies. 

\section{Simulation parameters}
We aim to perform a DNS\footnote{Direct Numerical Simulation}, which requires the grid resolution to be smaller than the Kolmogorov length scale $\eta$. We will use the pseudospectral method, meaning we must select the Reynolds number $Re$ such that the condition $k_{\text{max}} \eta = 1$ is satisfied. Here, $k_{\text{max}} = \frac{2\pi}{L} \frac{n_x}{2}$, where $n_x$ is the number of grid points along one spatial dimension, and $L$ is the domain size.

Additionally, we know the relationship $\frac{L}{\eta} = Re^{3/4}$, which links the Kolmogorov length scale $\eta$ to the Reynolds number $Re$. Using this information, we can express the Reynolds number as a function of the grid parameter $n_x$:

\begin{equation*}
k_{\text{max}} \eta = 1 \implies \frac{2\pi}{L} \frac{n_x}{2} \eta = 1
\end{equation*}

From this, we find:
\begin{equation*}
\frac{\eta}{L} = \frac{1}{\pi n_x}
\end{equation*}

Substituting $\frac{\eta}{L}$ into the expression $\frac{\eta}{L} = Re^{-3/4}$, we obtain:
\begin{equation*}
Re^{-3/4} = \frac{1}{\pi n_x}
\end{equation*}

Rearranging for $Re$, we find:
\begin{equation}
Re = \left( \pi n_x \right)^{4/3}
\end{equation}

This relationship ensures that the grid resolution is fine enough to resolve the smallest scales in the turbulence, satisfying the DNS requirement.