\documentclass[11pt,twocolumn]{article}
\usepackage[a4paper,left=2cm,right=2cm,includehead,nomarginpar,textwidth=10cm,headheight=14pt]{geometry}
\usepackage{graphicx}
\usepackage[utf8]{inputenc}
\usepackage[T1]{fontenc}
\usepackage{xcolor}
\usepackage{fancybox}
\usepackage{fancyhdr}
\usepackage{fourier-orns}
\usepackage{amsmath}
\usepackage{amsfonts}
\usepackage{amssymb}
\usepackage{caption}
\usepackage[english]{babel}
\usepackage[backend=bibtex,style=alphabetic]{biblatex}
\usepackage{pgfplots}
\usepackage{tikz}
\usepackage{tikz-3dplot}
\usepackage{cancel}
\usepackage{hyperref}
\usepackage{cleveref}
\usepackage{multicol}

\hypersetup{
    colorlinks=true,      
    linkcolor=blue,       
    urlcolor=blue,        
    citecolor=blue        
}

\newcommand{\pd}[2]{\frac{\partial #1}{\partial #2}}
\newcommand{\pdd}[2]{\frac{\partial^2 #1}{\partial #2^2}}

\renewcommand{\exp}[1]{\exp{\left( #1 \right)}}
\newcommand{\ex}[1]{\mathrm{e}^{#1}}

\addbibresource{references.bib}

\renewcommand{\headrule}{%
\vspace{-8pt}\hrulefill
\raisebox{-2.1pt}{\quad\decofourleft\decotwo\decofourright\quad}\hrulefill}

\begin{document}

\pagestyle{fancy}
\fancyhf{}
\fancyhead[LE]{\nouppercase{\rightmark\hfill\leftmark}}
\fancyhead[RO]{\nouppercase{\leftmark\hfill\rightmark}}
\fancyfoot[C]{\thepage}
\fancyfoot[LE,RO]{2024-2025}
\fancyfoot[RE,LO]{Rémy Guillermin}

\begin{titlepage}
   \begin{center}
       \vspace*{1cm}
		
		\Huge
       \textbf{Simulations of stratified rotating turbulence and comparison with oceanic wave turbulence}
            
       \vspace{1.5cm}
       
       \LARGE
       \textbf{Rémy Guillermin - supervised by Pierre Augier}
       
       \vfill
       
       \Large
       \textbf{Research Training III}
            
       \vspace{0.8cm}
     	
       
       \begin{center}
    	\begin{minipage}{0.49\textwidth}
        \centering
        \includegraphics[width=0.6\textwidth]{UGA-logo}
    	\end{minipage}
    	\begin{minipage}{0.49\textwidth}
        \centering
        \includegraphics[width=0.8\textwidth]{LEGI-logo}
    	\end{minipage}
	    \end{center}
		
		\vspace{0.8cm}
		
       2024-2025
            
   \end{center}
\end{titlepage}

\section*{Abstract}

In fluid mechanics, turbulent flows occur when the Reynolds number $\mathrm{Re} = UL/\nu \gg 1$, where $U$ and $L$ are the characteristic velocity and length, respectively, and $\nu$ is the kinematic viscosity. This condition is satisfied in cases such as the flow of an inviscid fluid\footnote{a fluid with no viscosity}, e.g., superfluids; when the flow is extremely fast, e.g., jet engine exhaust; or when the flow occurs at large scales, e.g., climate phenomena. 

When we look at large scale climate flows such as the ocean or the atmosphere, it is mandatory to consider two major changes in the basic Navier-Stokes equations: We need to account for the rotation of Earth as well as the stratification in density (ocean) or temperature(atmosphere). In this project, we have been focused on the effect of a density stratification of the flow, which will induce internal wave turbulence.

\section{Theoretical background}

The Navier-Stokes equations we will work with are 
\begin{subequations}
\begin{align}
	\pd{\rho}{t} + \vec{\nabla} \left( \rho \vec{u} \right) &= 0 \\
	\rho \left( \pd{\vec{u}}{t} + \left( \vec{u} \cdot \vec{\nabla} \right) \vec{u} \right) &= - \vec{\nabla} p - \rho g \hat{z} + \mu \Delta \vec{u}
\end{align}
\label{eq:NS}
\end{subequations}

It is useful to define the Froude number as $\mathrm{Fr}^2 = U^2/gH$ as the ratio of the advection term to the gravitational term. One must be familiar with the Boussinesq's approximation \cite{boussinesq_theorie_1897} in order to work with density stratified fluid. After applying this approximation on system \ref{eq:NS}, and after some rework, we find the wave propagation equation written as 
\begin{equation}
	\partial^2_t \vec{u} = \left( N^2 \nabla^2_h \nabla^{-2} \right) \vec{u} \label{eq:Wave Propagation}
\end{equation}
where $\nabla_h$ and $\nabla$ are the horizontal gradient operator and the spatial gradient operator, respectively, and $N^2 = -\frac{g}{\rho} \pd{\rho}{z}$ is the Brunt-Väisälä frequency \cite{pedlosky_geophysical_1979}. This equation allows us to find the dispersion relation of the gravity wave 
\begin{equation}
	\omega = \pm N \sin{\theta} \label{eq:Dispersion relation}
\end{equation}
where $\theta = k_z/\lvert \vec{k} \rvert$ is the angle between the wave-vector $\vec{k}$ and the z-axis. This means that $\omega$ is bounded inside the range $\left[ -N, +N \right]$. 

We will consider an incompressible flow ($\vec{\nabla} \cdot \vec{u} = 0$), which means that we can use the poloidal-toroidal decomposition \cite{schmitt_decomposition_1992} in the vertical direction. This allows us to express the kinetic energy $E_k$\footnote{In this paper, the specific kinetic energy $E_k/\rho$ will be named kinetic energy} as the sum of the poloidal and toroidal contribution $E_K = E_p + E_t$ where
\begin{equation*}
	E_{p,t} = \frac{1}{2} \int_{\mathbb{R}^3} \left( \vec{u}_{p,t} \right)^2 d^3 \vec{r}
\end{equation*}

We can also express the potential energy as
\begin{equation}
	E_\phi = \left\langle \frac{g^2 {\delta \rho}^2}{2 N^2 \rho_0^{\;2}} \right\rangle \label{eq:Potential Energy}
\end{equation}
where $\rho_0$ is the mean density of the fluid and $\delta \rho$ are the density perturbation along the density stratification $\rho_0 + \rho_b$.







\newpage

\printbibliography

\end{document}