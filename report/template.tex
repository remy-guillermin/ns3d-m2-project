%% 
%% Copyright 2007-2020 Elsevier Ltd
%% 
%% This file is part of the 'Elsarticle Bundle'.
%% ---------------------------------------------
%% 
%% It may be distributed under the conditions of the LaTeX Project Public
%% License, either version 1.2 of this license or (at your option) any
%% later version.  The latest version of this license is in
%%    http://www.latex-project.org/lppl.txt
%% and version 1.2 or later is part of all distributions of LaTeX
%% version 1999/12/01 or later.
%% 
%% The list of all files belonging to the 'Elsarticle Bundle' is
%% given in the file `manifest.txt'.
%% 
%% Template article for Elsevier's document class `elsarticle'
%% with harvard style bibliographic references

%\documentclass[preprint,12pt,authoryear]{elsarticle}

%% Use the option review to obtain double line spacing
%% \documentclass[authoryear,preprint,review,12pt]{elsarticle}

%% Use the options 1p,twocolumn; 3p; 3p,twocolumn; 5p; or 5p,twocolumn
%% for a journal layout:
%% \documentclass[final,1p,times,authoryear]{elsarticle}
%% \documentclass[final,1p,times,twocolumn,authoryear]{elsarticle}
%% \documentclass[final,3p,times,authoryear]{elsarticle}
%% \documentclass[final,3p,times,twocolumn,authoryear]{elsarticle}
%% \documentclass[final,5p,times,authoryear]{elsarticle}
 \documentclass[final,5p,times,twocolumn,authoryear]{elsarticle}

%% For including figures, graphicx.sty has been loaded in
%% elsarticle.cls. If you prefer to use the old commands
%% please give \usepackage{epsfig}

%% The amssymb package provides various useful mathematical symbols
\usepackage{amssymb}
\usepackage{lipsum}
%% The amsthm package provides extended theorem environments
\usepackage{amsthm}

%% The lineno packages adds line numbers. Start line numbering with
%% \begin{linenumbers}, end it with \end{linenumbers}. Or switch it on
%% for the whole article with \linenumbers.
\usepackage{lineno}

%% You might want to define your own abbreviated commands for common used terms, e.g.:
\newcommand{\kms}{km\,s$^{-1}$}
\newcommand{\msun}{$M_\odot$}

\journal{Graduate School@UGA}

\begin{document}

\begin{frontmatter}

%% Title, authors and addresses

\title{Simulations of stratified rotating turbulence and comparison with oceanic wave turbulence}

\author[first]{Guillermin Remy}

\begin{abstract}
%% Text of abstract
Example abstract for the Annals of Physics journal. Here you provide a brief summary of the research and the results.
\end{abstract}

%%Graphical abstract
%\begin{graphicalabstract}
%\includegraphics{grabs}
%\end{graphicalabstract}

%%Research highlights
%\begin{highlights}
%\item Research highlight 1
%\item Research highlight 2
%\end{highlights}

\begin{keyword}
%% keywords here, in the form: keyword \sep keyword, up to a maximum of 6 keywords
Numerical Simulation \sep Fluids Dynamic \sep Turbulence

%% PACS codes here, in the form: \PACS code \sep code

%% MSC codes here, in the form: \MSC code \sep code
%% or \MSC[2008] code \sep code (2000 is the default)

\end{keyword}


\end{frontmatter}

%\tableofcontents

%% \linenumbers

%% main text

\section{Introduction}
\label{introduction}

Here is where you provide an introduction to work and some background. For example building on previous work of image enhancement in optical astronomy \citep{vojtekova2021learning}, \cite{sweere2022deep} developed a method to improve the resolution of X-ray images from XMM-Newton to provide similar spatial resolution to Chandra.

\section{Title 2}
%%\label{}
\lipsum[1]

\subsection{Subsection title}

\begin{figure}
	\centering 
	\includegraphics[width=0.4\textwidth, angle=-90]{Annals_of_Physics_cover_image.pdf}	
	\caption{Annals of Physics journal cover} 
	\label{fig_mom0}%
\end{figure}

A random equation, the Toomre stability criterion:

\begin{equation}
    Q = \frac{\sigma_v \times \kappa}{\pi \times G \times \Sigma}
\end{equation}

\section{Title 3}
%%\label{}
\lipsum[2]

\subsection{Subsection title}
\lipsum[3]

\begin{table}
\begin{tabular}{l c c c} 
 \hline
 Source & RA (J2000) & DEC (J2000) & $V_{\rm sys}$ \\ 
        & [h,m,s]    & [o,','']    & \kms          \\
 \hline
 NGC\,253 & 	00:47:33.120 & -25:17:17.59 & $235 \pm 1$ \\ 
 M\,82 & 09:55:52.725, & +69:40:45.78 & $269 \pm 2$ 	 \\ 
 \hline
\end{tabular}
\caption{Random table with galaxies coordinates and velocities, Number the tables consecutively in
accordance with their appearance in the text and place any table notes below the table body. Please avoid using vertical rules and shading in table cells.
}
\label{Table1}
\end{table}


\section{Discussion}
%%\label{}
\lipsum[4]

\section{Summary and conclusions}
%%\label{}
\lipsum[1-4]


\section*{Acknowledgements}
Thanks to ...

%% The Appendices part is started with the command \appendix;
%% appendix sections are then done as normal sections
\appendix

\section{Appendix title 1}
%% \label{}

\section{Appendix title 2}
%% \label{}

%% If you have bibdatabase file and want bibtex to generate the
%% bibitems, please use
%%
\bibliographystyle{elsarticle-harv} 
\bibliography{references.bib}

%% else use the following coding to input the bibitems directly in the
%% TeX file.

%%\begin{thebibliography}{00}

%% \bibitem[Author(year)]{label}
%% For example:

%% \bibitem[Aladro et al.(2015)]{Aladro15} Aladro, R., Martín, S., Riquelme, D., et al. 2015, \aas, 579, A101


%%\end{thebibliography}

\end{document}

\endinput
%%
%% End of file `elsarticle-template-harv.tex'.